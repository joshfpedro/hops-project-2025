\documentclass[12pt,a4paper]{article}

% Language setting
\usepackage[british]{babel}

% Set page size and margins
\usepackage[a4paper,top=2cm,bottom=2cm,left=2.5cm,right=2.5cm,marginparwidth=1.75cm]{geometry}

%----------- APA style references & citations (starting) ---
% Useful packages
%\usepackage[natbibapa]{apacite} % APA-style citations.

\usepackage[style=apa, backend=biber]{biblatex} % APA 7th edition style citations using biblatex
\addbibresource{references.bib} % Your .bib file

% Formatting DOI in APA-7 style
%\renewcommand{\doiprefix}{https://doi.org/}

% Add additional APA 7th edition requirements
\DeclareLanguageMapping{british}{british-apa} % Set language mapping
\DeclareFieldFormat[article]{volume}{\apanum{#1}} % Format volume number

% Modify 'and' to '&' in the bibliography
\renewcommand*{\finalnamedelim}{%
\ifnumgreater{\value{liststop}}{2}{\finalandcomma}{}%
\addspace&\space}

%----------- APA style references & citations (ending) ---

\usepackage{amsmath}
\usepackage{graphicx}
\usepackage[colorlinks=true, allcolors=blue]{hyperref}
\usepackage{hyperref}
\usepackage{orcidlink}
\usepackage[title]{appendix}
\usepackage{mathrsfs}
\usepackage{amsfonts}
\usepackage{booktabs} % For \toprule, \midrule, \botrule
\usepackage{caption}  % For \caption
\usepackage{threeparttable} % For table footnotes
\usepackage{algorithm}
\usepackage{algorithmicx}
\usepackage{algpseudocode}
\usepackage{listings}
\usepackage{enumitem}
\usepackage{chngcntr}
\usepackage{booktabs}
\usepackage{subcaption}
\usepackage{authblk}
\usepackage[T1]{fontenc}    % Font encoding
\usepackage{csquotes}       % Include csquotes
\usepackage{diagbox}

% Customize line spacing
\usepackage{setspace}
\onehalfspacing % 1.5 line spacing

% Redefine section and subsection numbering format
\usepackage{titlesec}
\titleformat{\section} % Redefine section numbering format
{\normalfont\Large\bfseries}{\thesection.}{1em}{}

% Customize line numbering format to right-align line numbers
\usepackage{lineno} % Add the lineno package
\renewcommand\linenumberfont{\normalfont\scriptsize\sffamily\color{blue}}
\rightlinenumbers % Right-align line numbers

\linenumbers % Enable line numbering

% Define a new command for the fourth-level title.
\newcommand{\subsubsubsection}[1]{%
  \vspace{\baselineskip}% Add some space
  \noindent\textbf{#1\\}\quad% Adjust formatting as needed
}
% Change the position of the table caption above the table
\usepackage{float}   % for customizing caption position
\usepackage{caption} % for customizing caption format
\captionsetup[table]{position=top} % caption position for tables

% Define the unnumbered list
\makeatletter
\newenvironment{unlist}{%
\begin{list}{}{%
\setlength{\labelwidth}{0pt}%
\setlength{\labelsep}{0pt}%
\setlength{\leftmargin}{2em}%
\setlength{\itemindent}{-2em}%
\setlength{\topsep}{\medskipamount}%
\setlength{\itemsep}{3pt}%
}%
}{%
\end{list}%
}
\makeatother

% Suppress the warning about @parboxrestore
\pdfsuppresswarningpagegroup=1

%-------------------------------------------
% Paper Head
%-------------------------------------------
\title{Inoculum Dose, Diversity, Dispersal, and Damage: Simulating Optimal Economic Control of an Aerially-Dispersed Plant Pathogen at the Regional Scale}

\author[1]{Joshua F. Pedro}
\author[2]{Sharmodeep Bhattacharyya}
\author[1]{Shirshendu Chatterjee}
\author[3]{Thomas L. Marsh}
\author[4]{Jae Young Hwang}
\author[4,*]{David H. Gent}

\affil[1]{Department of Mathematics, City University of New York, NY 10031}
\affil[2]{Department of Statistics, Oregon State University, Corvallis, OR 97331}

\affil[3]{School of Economic Sciences, Washington State University, Pullman, WA 99163}
\affil[4]{U.S. Department of Agriculture-Agricultural Research Service, Forage Seed and Cereal Research Unit, Corvallis, OR 97331}

\affil[*]{Corresponding author: David H. Gent \texttt{dave.gent@usda.gov}}

\date{}  % Remove date



\begin{document}
\maketitle

\begin{abstract}
Plant pathogens that disperse by airborne propagules may cause damage that extends beyond the borders of individual fields. Developing sound management strategies therefore requires consideration of heterogeneity in pathogen transmission, the effectiveness of control measures, host susceptibility and pathogen virulence, and the resulting economic outcomes that scale up at the regional level with coordinated management. We use hop powdery mildew as a motivating pathosystem to develop a coupled epidemiological-economic model to enable simulation of the impact of epidemic conditions and coordinated management interventions on profitability. This pathosystem is a well-suited case study because disease development may be limited by primary inoculum and fungicide applications, yet the pathogen can spread via long-distance dispersal between fields and rapidly damage both crop yield and quality. We parameterized the model using data collected from a census sample of commercial hop yards in Oregon during 2014 to 2017, including the monthly incidence of plants with powdery mildew, fungicides applied by growers, and estimated revenue depending on how the incidence of diseased hop cones affects yield and the likelihood of crop devaluation. We show that conditions in the early stages of epidemics related to primary inoculum dose, pathogen diversity, and the intensity of management intervention interact and determine the optimal regional control strategy. As the likelihood of primary infection increases, due to either the dose of primary inoculum or virulence of the pathogen population, mean profitability decreases. These effects are most pronounced when primary infection occurs in yards most central in the network. The choice of how many fungicide applications to make in response to initial infection has little effect on profitability when primary inoculum is relatively infrequent. However, as primary inoculum increases, targeted fungicide applications made in the early stages of epidemics are essential for maximizing profitability region-wide. These principles hold across a range of market demand scenarios that change crop quality standards, even though relative profit losses increases under low demand conditions.  Our research provides a framework for formally understanding factors that influence the cost of disease in complex agricultural systems where pathogens disperse across management units.
\end{abstract}

\textbf{Keywords}: plant disease, pathogens, epidemiology, economic impacts, disease management, landscape-scale

\newpage

\tableofcontents

\newpage
%-------------------------------------------
% Paper Body
%-------------------------------------------
%--- Section ---%
\section{Introduction}

There is increasing scrutiny of pesticides in agriculture and a need to develop management strategies and policies that reduce inputs while maintaining productivity and profitability \parencite{Pimentel2005, Waard1993}. Disease management efforts are most often directed at the scale of individual fields or farms, but plant pathogens do not respect management units or jurisdictional boundaries. Management may be suboptimal or ineffective when not properly matched to the scale of pathogen dispersal \parencite{Gilligan2008, Gilligan2007, Irwin1999, Mahaffee2016, Thompson2016}. The relative importance of inoculum produced endogenously within a given field or exogenously in other locations may dictate the mitigation strategies and the need for collective action in the form of area-wide management \parencite{Bassanezi2013, garcia_ARP_2024, Filho2016, Lence2022}. 

Foundational theory in plant disease epidemiology suggests that mitigation efforts that reduce primary inoculum only delay epidemics, with the effect being inversely proportional to epidemic velocity \parencite{Plank1963}. Yet, a number of empirical studies and modeling suggest that early intervention is critical for containing the spread of disease outbreaks \parencite{Cunniffe2015b, Fraser2004, Severns2022}. Epidemics caused by pathogens capable of long-distance dispersal appear sensitive to the conditions in the initial outbreak area \parencite{Estep2014, Ojiambo2017, Severns2014, Severns2022, Xu1998}. Final epidemic severity may depend on the size of the initial focus, pattern of initial inoculum, proportion of susceptible hosts in the population, and connectivity to other susceptible hosts \parencite{Margosian2009, Pautasso2010, Severns2019}. The importance of these factors may depend sensitively on parameters controlling pathogen transmission and stochasticity in the earliest stages of an epidemic \parencite{Hartfield2013}. Concrete recommendations for control strategies also depend on the relative effectiveness and costs of controls.

The need for modeling to explicitly express processes and current understanding of a system is clear \parencite{Lofgren2014}, yet linking epidemiological models to crop damage and economic outcomes when pathogens potentially disperse over long distances is a challenging problem \parencite{Cunniffe2015a}. The motivating pathosystem for the research we present, hop powdery mildew, engenders many of these challenging aspects. The disease is caused by the fungus \textit{Podosphaera macularis} \parencite{Mahaffee2009} and remains one of the most difficult and costly problems affecting hop producers in the western U.S. and other production regions \parencite{Gent2008, Royle1978}. Success in breeding cultivars resistant to the disease have been met by emergence of strains of the fungus that can overcome host resistance when the said resistance is broadly deployed in the landscape \parencite{Block2021, Gent2017, Royle1978, Wolfenbarger2016}. Therefore, management still relies heavily on cultural practices that reduce inoculum density and fungicides to slow the rate of disease progression \parencite{Gent2019b, Mahaffee2009, Nelson2015, Probst2016, Royle1978}. This pathosystem has several biological and economic attributes that make it an exemplar for developing the modeling framework we propose herein, and also broadly of interest for other aerially-dispersed plant pathogens in complex social-ecological systems.

First, the causal pathogen is an obligate biotroph with a host range limited to the plant family Cannabanaceae \parencite{Mahaffee2009}, with hop being the dominant and most important host cultivated at the time of these studies \parencite{Rivedal2023}. Consequently, it is possible to infer disease transmission between yards because other sources of inoculum are trivial. Further, this system is amendable to modeling dispersal between yards as \textit{P. macularis} exhibits annual cycles of emergence, colonization, and extinction. The fungus persists from season to season in the Pacific Northwestern U.S. only in association with living host tissue, provisioned by infected crown buds due to the absence of the sexual stage of the pathogen in this region \parencite{Gent2008, Weldon2021, Wolfenbarger2015}. Bud infection and overwintering of the pathogen leads to highly focal infections owing to the emergence of heavily infected shoots in spring, the so-called flag shoots, which occur at a low frequency in commercial hop yards \parencite{Gent2019b,Laurie2023}. From these focal infections, \textit{P. macularis} disperses via wind and readily spreads within the network of hop yards to create regional epidemics \parencite{Gent2019a}. At the end of the cropping season, the pathogen becomes locally extinct in most yards due to its low overwintering survival \parencite{Gent2019b, Gent2018}. Thus, a new realization of the disease development and transmission process can be observed annually.

Pathogenic variation is common in pathogens across plant and animal hosts \parencite{Burdon1997, Koelle2022} and has important implications for the dynamics of disease transmission and severity \parencite{Mundt2002, Ostfeld2012}. This is especially relevant in agricultural systems where \textit{R}-gene-mediated resistance is deployed and creates a mosaic of resistant and susceptible hosts \parencite{McDonald2002}. Host resistance to powdery mildew is commercially available in hop and multiple pathogenic variants (strains) exist in \textit{P. macularis} in certain populations \parencite{Gent2017, Royle1978, Wolfenbarger2016}. As hop plants are long-lived perennials \parencite{Neve1991}, relatively stable patterns of resistant and susceptible cultivars are present in the landscape.

The pathosystem is further of interest because powdery mildew may cause economic losses by reducing both yield and potentially crop quality \parencite{Gent2014, Neve1991, Royle1978}. This is a common scenario in agricultural systems where crop damage is not attributed solely to direct losses in yield, but additionally or primarily from reductions in crop quality \parencite{Savary2006}. This is particularly important for crops marketed directly to consumers with cosmetic value \parencite{Esker2013, Zadoks1985}. In these situations, the financial loss incurred by the grower may be linked in complex ways to market demand \parencite{Zadoks1985}. Indeed, this is the situation with hops and powdery mildew because the disease may reduce brewing value and cause conspicuous visual defects from degraded cone color. For this reason, growers apply fungicides repeatedly during the season with the goal of minimizing inoculum pressure for the critical cone phase of the disease \parencite{Nelson2015, Royle1978}. Fungicide applications for powdery mildew may occur over a period of multiple months to minimize foliar infections in spring and then later in summer to minimize cone infection \parencite{Nelson2015, Twomey2015}. Thus, growers must make multiple decisions on when to begin treatment and how intensively to treat.

Adding even more uncertainty, hops are sold almost entirely through marketing contracts that stipulate quality standards, usually with vague or subjective quality standards. Contracting is conducted for numerous agricultural commodities, and in 2017 21 percent of total U.S. crop production was under a contract agreement \parencite{MacDonald2019}. Price risk reduction is stated as a major incentive for contracting \parencite{Key2005}. The potential for crop devaluation or, in the worst case rejection, may substantially affect disease management choices since the producer assumes firstly the risk of crop damage and secondly the penalty for failing on a contractual obligation \parencite{Zhang2015}. We expect that optimal disease control strategies may therefore be sensitive to contract structure when crop value or saleability is inseparably linked to crop quality metrics. Furthermore, the optimal control strategy may depend on practices in other fields and other farms given the potential for long-distance dispersal of the pathogen.

Motivated by this pathosystem, we draw upon an exceptional rich data set collected from a census sample of hop yards in Oregon over a four-year period for our analyses \parencite{Gent2019a}. We formulate an epidemiological model for the development and spread of powdery mildew at the regional level, including the effect of fungicides applied in a field of interest and all other potential source fields. We couple the epidemiological model to an economic model of expected revenue and costs associated with disease management, devaluation or rejection due to quality defects, and market conditions. We then simulate varying conditions of the initial phases of epidemics related to primary inoculum dose (frequency), pathogen diversity, centrality in the transmission network of fields, and management intervention to identify control strategies that maximize profit under varying market demand scenarios.

\section{Material and methods}

\subsubsection{Study system}

Our study system is the hop production region in western Oregon. Oregon is one of the leading hop producing regions in the U.S. and commercial production is concentrated in a few counties in the Willamette Valley in the western portion of the state \parencite{Barth1994}. Hop powdery mildew was first confirmed in the field in Oregon in 1998 and has occurred annually each year since then \parencite{Mahaffee2003, Ocamb1999}. Historically, powdery mildew has tended to occur most regularly and most severely in production regions in the eastern extent of the Willamette Valley, presumably due to the frequency of overwintered inoculum \parencite{Gent2019b, Turechek2001} and subsequent dispersal from the resultant disease foci \parencite{Gent2019a}. We intentionally focused on hop farms in the eastern production regions for data acquisition. The diversity of cultivars and production practices in this region are summarized elsewhere \parencite{Laurie2023}.

\subsubsection{Biological data acquisition}

We obtained monthly data on the incidence of plants with powdery mildew or primary infection (i.e., occurrence of a flag shoot) from a census survey of hop yards. Disease assessments were conducted monthly from April to July during each of 2014 to 2017. For brevity, a summary of the disease assessments methods is provided here; a complete description of the methods are given in \parencite{Gent2019a}. There were 8 to 10 farms sampled per year, all within Marion County, Oregon, with a maximum distance between yards of 26 km. After cleaning, data were available for 99 yards assessed in 2014, 113 in 2015, 116 in 2016, and 122 in 2017. All cultivars were evaluated, independent of their susceptibility to powdery mildew \parencite{Laurie2023}. The incidence of plants with powdery mildew was assessed using a modification of cluster sampling methods described previously \parencite{Turechek2001, Turechek2004}. 

As noted previously, there are multiple pathogenic strains of \textit{P. macularis} that differ in their ability to cause disease on hop plants with specific resistance genes (\textit{R}-genes). Two strains of the pathogen were relevant at the time of the field surveys and are differentiated based on their ability to cause disease on plants with a resistance termed \textit{R6} \parencite{Padgitt-Cobb2019, Wolfenbarger2016}. The presence of \textit{R6} resistance is analogous to immunization that provides complete, but strain-specific, protection from disease. The presence of \textit{R6} in hop plants provides resistance to powdery mildew only when the pathogen lacks a corresponding virulence (\textit{V}). For shorthand, we refer to strains that cannot infect plants with \textit{R6} resistance as non-\textit{V6}-virulent. Other strains of the pathogen may infect susceptible plants that lack \textit{R6}, but importantly, can also infect plants that possess \textit{R6}. These strains are dubbed \textit{V6}-virulent. \textit{V6}-virulent strains are promiscuous in the yards they may affect, as they infect hop cultivars that both possess or do not possess \textit{R6}, analogous to a vaccine-evading strain of a pathogen.

When we detected powdery mildew in a given hop yard it was necessary to match the strain of the pathogen present to potential yards where the pathogen could disperse to cause disease. Therefore, the initial strain of the pathogen present was determined to be \textit{V6}-virulent or non-\textit{V6}-virulent using bioassays as described previously \parencite{Gent2019a, Wolfenbarger2016}.

\subsubsection{Pesticide use and costs}

We obtained pesticide application records from each grower for all yards sampled during 2014 to 2017 and interrogated their records to determine the timing and dosage of applications of herbicides, fungicides with activity against powdery mildew, and adjuvant additives. We did not consider insecticides or miticides because these were not directly relevant for the present analysis of powdery mildew. We then estimated the January 2022 real costs of the relevant pesticides and adjuvants applied by requesting price quotes for each product from each of three vendors in western Oregon that service hop producers as described by \parencite{hwang_what_2024}. We estimated real prices using January 2022 as the base by adjusting the nominal price by the producer price index for farm products for each year available from the U.S. Bureau of Labor Statistics, and then averaged over all available years to derive a single real price per unit. 

\subsection{General modeling approach}

We introduce a linked epidemiological-economic statistical model of epidemic development and apply this model to hop powdery mildew to simulate economic outcomes due to disease management costs and crop damage from direct losses in yield and quality defects. The epidemic model contains both stochastic and deterministic components to estimate development of the foliar phase of powdery mildew originating from initial occurrences of \textit{P. macularis} via bud perennation, autoinfection at the scale of individual hop yards from inoculum endogenous to a given yard, and exogenous inoculum dispersed from other yards in the region. The population moment estimate of the incidence of plants with powdery mildew is derived from a function of the probability of plant being diseased, as moderated by fungicide use. Profits are the summation of revenue and fixed and variable costs as influenced by fungicide inputs, and yield and crop devaluation due to quality defects as a function of disease incidence. External drivers are inputs of wind data, fungicide use, and initializing values of the probability of initial (primary) infection due to bud perennation, expected crop yield in the absence of disease, and price per unit of yield. The model is spatially explicit at the scale of individual hop yards and uses the actual location and size of hop yards in the data set for a representative year; there is no attempt to account mechanistically for focus development or disease spread within hop yards.

We use the linked models to simulate profit levels resulting from the varying epidemic conditions. Concretely, these conditions were related to initial infection frequency, pathogenic diversity of the initial strain of the pathogen, the location of the initial infections, and the number of fungicide applications made in the first month of the epidemic. We aggregate responses from individual yards to scale to the regional population of yards. We then ran each simulation scenario under three market demand scenarios to understand the sensitivity of the results to market conditions.

\subsection{Submodels}

\subsubsection{Epidemic network model}

An individual hop yard is considered a node in a network of yards in the spatial extent of interest. Disease status of a yard in a given month is a nonlinear function of its disease incidence in the preceding month, susceptibility to two races of \textit{P. macularis}, and disease spread from other nodes as influenced by their disease incidence and area (source strength), distance apart, and wind run in the preceding month \parencite{Gent2019a}. We expanded this model by introducing two parameters that moderate disease in a given hop yard and disease spread from other hop yards based on the number of fungicide applications made in the prior month. We also generalized the model by expressing the dispersal kernel as a function of distance and source strength to accommodate various functional forms. The impact of exogenous inoculum depends jointly on the pathogen virulence in the source yards and susceptibility of the cultivar planted in the target yard.

%In the expanded model, we introduced s_i and s_j which moderates disease in target yard {i} and potential source yards {j} based on the number of fungicide sprays made in each. The number of diseased plants is given by....
The number diseased plants is given by $Y_{i}$, among a sample of $n_{i}$ plants, with $i = 1,\ldots,N$ indexing yard identity. The probability of $Y_{i}$, is taken as binomially distributed with the predictor function expressed on the log-odds scale as:

\begin{equation} \label{eq:1}
\log \left(\frac{p_i}{1-p_i}\right)=\beta+\delta\left(\frac{\tilde{y}_i}{n_{\tilde{y}_i}} \exp {\left(-\eta_1 s_i\right)}\right)+\gamma \sum_{j=1}^{M_i}\left(\frac{a_j z_j}{n_{z_j}} w_{i j} \exp {\left(-\eta_2 s_j\right)} f {\left(d_{i j}; \alpha\right)}\right)
\end{equation}
\\
Covariates and parameter interpretations are given in Tables \ref{tab:table1} and \ref{tab:table2}. Parameters are allowed to vary for time transition periods from May to June and June to July. Equation \ref{eq:1} defines the model for disease development on foliage at the plant scale.  We explored an exponential \parencite{Gent2019a} and also a power-law function for the dispersal kernel,  but present results only for the power-law function given theoretical considerations of expected dispersal characteristics of \textit{P. macularis} \parencite{Severns2019}


\begin{table}[ht]
\centering
\caption{Covariate notations for each yard of interest $i=1,\ldots, N$ in a particular month, and all other yards $j=1,\ldots,M_i$ relative to yard $i$.}
\label{tab:table1}
\begin{tabular}{ll}
\hline Variable & \multicolumn{1}{c}{Description} \\
\hline$\tilde{y}_i$ & Number of diseased plants at yard $i$ in the prior month \\
$n_{\tilde{y}_i}$ & Number of plants sampled at yard $i$ in the prior month \\
$a_i$ & Area (hectares) of yard $i$ \\
$z_j$ & Number of diseased plants at yard $j$ in the prior month \\
$n_{z_j}$ & Number of plants sampled at yard $j$ in the prior month \\
$a_j$ & Area (hectares) of yard $j$ \\
$d_{ij}$ & Distance (kilometers) from centroids of yard $i$ to yard $j$ \\
$w_{ij}$ & Wind vector on $i-j$ direction in the prior month \\
$s_i$ & Number of fungicide sprays in yard $i$ in prior month \\
$s_j$ & Number of fungicide sprays in yard $j$ in prior month\\
\hline
\end{tabular}
\end{table}

\\



\begin{table}[ht]
\centering
\caption{Interpretation of model parameters.}
\label{tab:table2}
\begin{tabular}{lp{0.7\textwidth}}
\hline
Parameter & \multicolumn{1}{c}{Interpretation} \\
\hline
$\beta$ & Baseline log-odds of disease, after accounting for autoinfection and dispersal.\\
$\delta$ & Change in log-odds of disease associated with autoinfection at the yard scale, after accounting for disease spread.\\
$\gamma$ & Distance-adjusted change in log-odds of disease associated with disease spread from other yards, after accounting for autoinfection.\\
$\alpha$ & Dispersal parameter providing distance adjustment to change in log-odds of disease associated with individual source yards. \\
$\eta_1$ & Change in log-odds of disease associated with autoinfection at the yard scale, after accounting for fungicide sprays.\\
$\eta_2$ & Dispersal parameter providing fungicide spray adjustment to change in log-odds of disease associated with individual source yards. \\
\hline
\end{tabular}
\end{table}

\subsubsection{Economic model}

The profit function $\Pi$ at time $t$ over the growing season ($t=0,...,T$) measures the total profit per hectare across all hop yards $i = 1 \ldots N$ 

$$
\Pi(s_{i t}) = \sum_{i=1}^N (\pi_{i t})
$$
where $\pi_{it}$ is the profit per hectare at time $t$ for yard $i$. This profit equation is conditioned on the number of  fungicide sprays $s_{it}$  across the season and subject to constraints imposed by the epidemic network model in Equation \ref{eq:1} and by the price-quality relationships detailed below. Here, we assume risk neutrality of growers.

%$$
%\log \left(\frac{p_{i t}}{1-p_{i t}}\right)=\beta_t+\delta_t\left(\frac{\tilde{y}_{i t}}{n_{\tilde{y}_{i t}}} \exp {\left(-\eta_{1 t} s_i\right)}\right)+\gamma_t \sum_{j=1}^{M_i}\left(\frac{a_j z_{j t}}{n_{z_{j t}}} w_{i j t} \exp {\left(-\eta_{2 t} s_{j t}\right)} f\left( d_{i j}; \alpha_t\right)\right)
%$$

The profit  $\pi_{i t}$ for each yard $i$ at time $t$ is defined as the difference between revenue and cost for each yard at time $t$:

$$
\pi_{i t} = R_t(q_i,v; p_i) - C_i(s_{i t})
$$
Revenue is only realized in the final period \textit{T}, so that revenue is defined by $R_i(q_i,v; p_i) = R_T$, otherwise $R_i(q_i,v; p_i) = 0$. Revenue is a function of both yield $q_i$ and price $v$, conditioned on the disease probability level $p_i$.  Costs $Ci$, defined in more detail below, are a function of fixed costs and the number of fungicide sprays, $s_{it}$.

    Both yield quantity and price are adjusted by the incidence of cones with disease and associated quality defects as inferred from cone color. We used relationships reported in the literature to link the foliar phase of the disease to incidence of cones with powdery mildew yield, and defects to cone color. First, the incidence of leaves with powdery mildew was estimated from the incidence of plants, $p_i$, with powdery mildew through the hierarchical relationship between disease at these scales \parencite{Turechek2004}.
$$\text{Leaf Incidence} = 1-(1-p_i)^{D/n}$$
where $n$ is the number of leaves sampled per plant and $D$ is the index of dispersion. We assume \textit{n} = 50 and \textit{D} as reported previously \parencite{Turechek2004} The incidence of cones with powdery mildew was assumed to be a linear function of the incidence of leaves with powdery mildew \parencite{Turechek2001} as:
$$
\text{Cone Incidence} = a_1 \times \text{Leaf Incidence} + b_1
$$
Yield loss was then estimated assuming a linear relationship with cone incidence, which is in turn derived from leaf incidence \parencite{Gent2014}:
$$
\text{Yield Loss} = a_2 \times \text{Cone Incidence}
$$
We considered cone color a surrogate for quality defects associated with powdery mildew that could lead to crop devaluation or rejection based on production contract standards. Cone color is modeled using an exponential decay function \parencite{Gent2014}:
$$
\text{Cone Color} = 10 + a_3 \cdot (1 - \exp{\left(-b_3 \times \text{Cone Incidence}\right)})
$$
These empirical relationships are presented in Figure \ref{fig:fig1}.

\begin{figure}[h]
    \centering
    \includegraphics[width=0.8\textwidth]{figures/fig1.png}
    \caption{Derived relationship between the incidence of hop plants and cones with powdery mildew and subsequent cone color and yield damage. A, Relationship between the incidence of plants with powdery mildew and the incidence of leaves with powdery mildew. B, Estimated relationship between the incidence of leaves with powdery mildew and subsequent incidence of hop cones with powdery mildew. C, Derived relationship between the incidence of plants with powdery mildew and the incidence of cones with powdery mildew based on relationships in A and B. D, Estimated relationship between the incidence of cones with powdery mildew and cone color as assessed on a 10-point scale. E, Estimated relationship between the incidence of cones with powdery mildew and proportion of yield damage. F, Final derived relationship between the incidence of plants with powdery mildew and proportion of yield damage based on relationships in A, B, C, and D.}
    \label{fig:fig1}
\end{figure}

Market conditions and production contracts also impact revenue.  Production contracts were reviewed and plausible damage functions were derived from expert opinion for crop devaluation from moderate damage, salvage value if the crop was severely affected and was alternatively extracted for alpha-acids, or unsaleable due to extreme damage. We modeled these scenarios as a sigmoid function with varying thresholds dependent on low, medium, and high market demand (Figure 2), as: 
%add market demand conditions for figure 2
$$
\text{Adjusted Price} = \text{Initial Price} \times \sigma(\theta_0 + \theta_1 \cdot \text{Cone Color})
$$
where $\sigma$ is the sigmoid function and $\theta_0, \, \theta_1$ are parameters estimated for different market demand scenarios. Revenue in a given yard is then calculated as the product of Adjusted Price and expected yield. Yield in the absence of disease was taken as the average yield reported in 2020 by the National Agricultural Statistics Service for the four cultivars we considered, as we describe below. Initial price was taken from representative production contracts for each of the three estimated levels of demand.

The cost function for each hop yard $C_i$ incorporates the costs associated with fungicide sprays along with fixed costs and other variable costs over the production season,

$$
C_i(s_{i,t}) = C_F + C_V + \sum_{t=0}^T (C_{s,t} + C_{a,t})
$$
where $C_F$ is the fixed cost, $C_V$ represents other variable costs, $C_{s,t}$ is the fungicide cost, and $C_{a,t}$ is the application cost at period $t$. Fixed and other variables costs were obtained from Galinato (2020). 

\subsubsection{Parameter estimation}

The parameters in the epidemic network model were estimated using maximum likelihood. The likelihood function was constructed by assuming the number of diseased plants $Y_i$ in each yard $i$ follows a binomial distribution with probability $p_i$ given by the inverse logit of $\eta_i$ in Equation \ref{eq:1}. The log-likelihood of the parameters $\boldsymbol{\theta} = (\beta, \delta, \gamma, \alpha, \eta_1, \eta_2)$ given the observed data $\mathcal{D} = {(y_i, n_i, \tilde{y}_i, n_{\tilde{y}_i}, a_i, s_i, {z_j, n_{z_j}, a_j, d_{ij}, w_{ij}, s_j})}$ where $i=1 \ldots N$ and $j=1 \ldots M_i$ is:
$$
\ell(\boldsymbol{\theta}; \mathcal{D}) = \sum_{i=1}^N \left[ y_i \log p_i + (n_i - y_i) \log (1-p_i)\right]
$$
where $p_i = \text{logit}^{-1}(\eta_i)$ and $\eta_i$ is a function of $\boldsymbol{\theta}$ and the observed covariates.
The maximum likelihood estimator $\hat{\boldsymbol{\theta}}$ was obtained by maximizing $\ell(\boldsymbol{\theta}; \mathcal{D})$ subject to bounds on the parameters to ensure they were biologically plausible. This was done using the Sequential Least Squares Programming (SLSQP) algorithm, a gradient-based nonlinear constrained optimization method \parencite{Kraft1988}. Analytical expressions for the gradient and Hessian of the negative log-likelihood were derived and provided to the algorithm to improve convergence. Multiple random initializations of the parameters were used to ensure a global optimum was reached. The model was fit using data for all years, but separately for the May to June and June to July transition periods to allow parameters to vary between these periods. Convergence was assessed by monitoring the magnitude of the largest gradient and the positive definiteness of the Hessian matrix at the obtained solution.
All analyses were conducted in Python, making use of the NumPy, SciPy, and Statsmodels libraries for numerical computing and statistical modeling.


%OLD VERSION
% \subsection{Simulation experiments}

% For the simulation experiments described below, we first created a synthetic landscape consisting of the actual location of the yards in each of the years 2014 to 2017. We randomly assigned the cultivars Chinook, Simcoe\textsuperscript{\textregistered}, Nugget, and Mosaic\textsuperscript{\textregistered} in equal proportions to each of the farms in the landscape. We selected these cultivars as representatives in our simulations because they were: (i) commercially relevant at the time of this research; (ii) can be sold for direct use in brewing or alternatively can be processed to extract alpha-acids as a secondary market; and (iii) because Nugget and Mosaic\textsuperscript{\textregistered} possess \textit{R6} whereas Chinook and Simcoe\textsuperscript{\textregistered} do not. We assumed equal susceptibility to powdery mildew given a compatible strain of pathogen. We randomly placed these cultivar on each hop farm in equal proportions to create a patchwork of yards that did or did not possess \textit{R6}. In all simulations we initiated epidemics in May assuming initial infection (due to occurrence of a flag shoot) was a binomial process, varying factors related to the initial conditions of the disease outbreak. Resulting disease levels in each month were then substituted into the epidemic model to estimate the number of infections in the next month, eventually leading to the final period when incidence of diseased cones and crop damage was estimated. 

% We varied the number of fungicide applications for $t=0$ to be between 0 and 12 if disease was present, otherwise, if a yard did not contain disease then we used a default of 0.43 fungicide applications, which was the average number of sprays for yards with no disease in the raw data. Cost of that application was estimated from the average fungicide application costs in the data set. The number of fungicide applications $s_{i,t}$ for $t>0$  is modeled as a function of disease incidence in the previous period, using separate Poisson regression models for different time periods:
% $$
% s_{i,t} \sim \text{Poisson}(\lambda_{t}(p_{i,t-1}))
% $$
% where $\lambda_{t}$ is the expected number of sprays at period $t$ given disease incidence $p_{i,t-1}$ in the previous period. Models were fit using observed disease incidence and fungicide applications reported by the growers in the data set.

% In each scenario, we ran 100 simulations, without re-randomization of the placement of cultivars on each farm for each scenario. We calculated the mean of the relative change in profit per hectare for each simulated scenario, relative to a baseline condition of no disease and no fungicide applications. We conducted these simulations for each of the three levels of market demand and for each year. 

% \subsubsection{Primary inoculum dose}

% We first simulated the impact of primary inoculum dose on epidemic development. For this simulation we assumed the initial strain of the pathogen was only \textit{V6}-virulent such that all varieties could potentially be infected. We assumed the probability of a plant having primary inoculum was a binomial process with random placement. We considered four levels of initial probability of disease 0.0001, 0.001, 0.01, and 0.1, a large but plausible range of primary infections levels \parencite{Laurie2023}.  % mention that the data set was close to XXX in initial period over the years (give range and average).


% \subsubsection{Pathogen diversity}

% We next varied the proportion of initial frequency of \textit{V6} and non-\textit{V6}-virulent pathogen strains in the synthetic landscape to understand the impact of pathogen diversity. We varied the proportion of initial infections caused \textit{V6} or non-\textit{V6}-virulent strains by randomly assigning 0\%, 25\%, 50\%, 75\% and 100\% of the yards the \textit{V6}-virulent strain for each of the four levels of initial probability of disease as described previously. % mention the range of v6 in the data set over the years

% \subsubsection{Primary inoculum location}

% We varied the location of primary inoculum to understand the importance of where primary inoculum occurs. For all yards in the region, we constructed a fully connected graph where the nodes were the locations of the yards with edges weighted by $a_i \times w_{i j} \times \frac{1}{(1+d_{ij})^2}$. These edge weights represent transmission potential should disease occur \parencite{Gent2019a}.  We computed the weighted degree centrality of the yards, ranked each yard based on its degree-centrality,  and placed primary inoculum in each decile. For each decile, we varied primary inoculum dose and pathogen diversity as described previously.  
%mention figure that shows degree centrality map here

%UPDATED VERSION
\section{Simulation experiments}
\label{sec:simulation_experiments}

For the simulation experiments described below, we constructed a synthetic landscape using the actual locations and sizes of hop yards in each of the years 2014 to 2017 from our data set. Each hop yard was planted to one of four representative cultivars (Chinook, Simcoe\textsuperscript{\textregistered}, Nugget, or Mosaic\textsuperscript{\textregistered}), chosen because they were commercially relevant at the time of this research, either possessed (\emph{R6}) or did not possess the \emph{R6} resistance, and can be sold for direct use in brewing or alternatively can be processed to extract alpha-acids as a secondary market. We assumed equal susceptibility to powdery mildew given a compatible strain of pathogen. Cultivar assignments were made at random for each yard, subject to a 1:1 ratio of cultivars that possess \emph{R6} or\ non-\emph{R6} in the entire landscape. The same yard-level assignments were used across simulation runs for a given year. 

We initiated epidemics at the start of May under varying scenarios by specifying an \emph{initial probability of disease}, which governed the expected frequency of initial infections (flag shoots) in the population of hop yards. In reality, any given yard may or may not harbor flag shoots, making the true distribution of early-season disease unknown. We therefore modeled the occurrence of initial infections stochastically as follows. For each yard in the landscape, we first identified the proportion of plants there that were susceptible, which depends on whether a yard was planted to a cultivar susceptible to \emph{V6}-virulent and/or non-\emph{V6}-virulent strains. We also identified which quantile of the degree-centrality distribution the yard occupied, where degree-centrality weights each yard by the product of yard area, average wind run, and a decreasing function of inter-yard distance as a measure of disease transmission potential \parencite{Gent2019a}. Denote this quantile by $q_i$ for yard~$i$, and let $0 \le q_i \le 1$ (discretized into 20\% quantile bins).

Next, we specified a desired mean initial probability of disease across the \emph{entire} landscape, denoted by $p_0$. Because yards differ in their centrality and susceptibility, we introduced an adjustment factor so that the final landscape-wide fraction of plants with flag shoots would closely match~$p_0$. Specifically, within each centrality quantile $q$, let:
\[
\text{(proportion susceptible in quantile } q) = \mathcal{S}_q, 
\quad
\text{(proportion of yards in quantile } q) = \mathcal{P}_q.
\]
We then scaled $p_0$ by $\frac{1}{\mathcal{S}_q}\,\frac{1}{\mathcal{P}_q}$, effectively concentration flag shoots into those yards that were actually susceptible in the particular quantile~$q$. Denote this adjusted probability by $p_0^{\ast}(q)$. For each yard~$i$ in quantile~$q$, we then drew:
\[
y_{0,i} \;=\; \mathrm{Binomial}\bigl(n_i,\;p_0^{\ast}(q)\bigr),
\]
where $n_i$ is the number of plants in yard~$i$. Thus, $y_{0,i}$ is the number of initial infections (flag shoots) in yard~$i$ at the start of May. $y_{0,i}$ was set to zero if the initial strain that was randomly placed was biologically impossible with the cultivar in that yard.  

With these initial conditions set, we simulated disease progression through May, June, July, and into the early cone-development period using the network-based epidemiological model described in Section~\ref{sec:model-description}. In each monthly step, the incidence of infected plants in yard~$i$ was predicted using the fitted coefficients from Equation~\eqref{eq:1}. Fungicide applications served as an input to the model via coefficients that reduce the rate of new infections in both the focal yard and the potential source yards; the number of sprays in a yard was itself predicted by a Poisson regression fitted from the data on how many fungicide applications growers made in a yard in response to the incidence of diseased plants in the previous month (see Section~\ref{sec:economic_model}). In May, we specified that if a yard did not contain disease then it still received a baseline of 0.43 fungicide applications, which was the average number of sprays for yards with no disease in the raw data. We additionally allowed a user-defined number of early sprays in May for yards that \emph{did} harbored at least one flag shoot (i.e.\ $y_{0,i} > 0$). These early sprays could range from zero, no additional sprays beyond the baseline, up to 10 sprays to represent an extremely aggressive early-season spray strategy. 

After simulating disease progress through the end of July, we computed the incidence of diseased cones using the relationship in Figure \ref{fig:fig1}C. The economic model then used these incidence levels to estimate yield losses, potential quality reductions (via the cone-color model), and resulting net returns after accounting for fixed costs, variable costs, and the cost of sprays. We repeated these simulations for a range of $p_0$ values, different fractions of \emph{V6}-virulent strains present in the initial inoculum, and different yard-level centralities at which the initial flag shoots occurred. For primary inoculum dose, we varied $p_0$ from 0.00001 to 0.01, a large but plausible range of primary infections levels \parencite{Laurie2023}. The proportion of initial infections caused \textit{V6} or non-\textit{V6}-virulent strains were varied from 0\% to 100\% in 25\% steps. For degree centrality, we placed the primary inoculum in into each 20\% quantile. Unless otherwise noted, each scenario was run 100 times to capture variability in binomial draws. We report either the mean or percentiles of the \emph{relative} change in profit per hectare compared to a no-disease/no-spray baseline. 


%summary table begin
\subsection{Summary of simulation parameters}
\label{sec:simulation_parameters_table}

Table~\ref{tab:summary_parameters} provides a concise overview of the key parameters we varied in the nested \texttt{for}-loops of our simulation experiments (Section~\ref{sec:simulation_experiments}). Each parameter name is kept brief so it fits on a single line in the table. For every combination of these parameter values, we ran the simulation for each of the four years (2014--2017) and for each of three market-demand scenarios (low, moderate, high).

\begin{table}[H]
\centering
\caption{Key parameters varied in the simulation, their tested ranges, and the rationale or source.}
\label{tab:summary_parameters}
\begin{tabular}{p{3.0cm} p{3.7cm} p{7.3cm}}
\toprule
\textbf{Parameter} & \textbf{Values Tested} & \textbf{Rationale / Data Source} \\
\midrule

Year 
& 2014, 2015, 2016, 2017 
& Real yard size, distance and wind patterns were available for each year, reflecting distinct conditions.\\[6pt]

Simulations 
& 100 
& Each scenario is replicated to capture variability from binomial draws of initial disease.\\[6pt]

Quantiles
& 5 
& Yards are classified into five bins by degree-centrality, used when varying location-based initial infections.\\[6pt]

Pathogen-diversity
& 0.0, 0.25, 0.50, 0.75, 1.0 
& Varies fraction of initial inoculum from \emph{V6}-virulent vs.\ non-\emph{V6}-virulent strains.\\[6pt]

Sprays in May 
& 0, 1, 2, 3, 4, 5, 6, 7, 8, 9, 10 
& Extra sprays (beyond baseline) if a yard has flag shoots. Upper end (10) tests diminishing returns.\\[6pt]

Initial infection probability 
& 0.00001, 0.00005, 0.0001, 0.0005, 0.001, 0.005, 0.01 
& Represents the probability of a plant harboring overwintered infection. Spans extremely rare (e.g.\ 0.00001) to modestly frequent (e.g.\ 0.01).\\[6pt]

Market scenario 
& low, moderate, high 
& Captures three contrasting demand/price structures, impacting crop-value and quality penalties.\\

\bottomrule
\end{tabular}
\end{table}

Beyond these parameters, the simulation code uses yard-level attributes (area, distance matrices, wind data) and the fitted model coefficients described in Sections~\ref{sec:model-description} and \ref{sec:economic_model}. In each scenario, the model outputs include time-series of disease incidence and final net profit estimates under the specified market-demand conditions, allowing for comparisons of how early-season sprays and pathogen diversity affect economic outcomes.
%table end


\section{Results}

We simulated epidemics that varied in the initial conditions related to the dose of primary inoculum, diversity of the pathogen, and location of primary inoculum to identify coordinated management responses that maximize profit at the regional scale. We address each aspect of the simulated scenarios in the sections that follow. We present data from only 2017 as our exemplar, as results were qualitatively similar in all years (Appendix Figures S1 to S9).
c
%insert few sentences introducing parameter estimates and refer to table on parameter estimates.
\subsubsection{Parameter estimation}

\subsubsection{Primary inoculum dose}

There was a pronounced effect of primary inoculum dose on profitability and the optimal number of fungicide applications, as shown in the bottom row of Figure X. When primary inoculum was infrequent (initial probability 0.0001), making no sprays or treating stringently when disease occurred (up to 10 times in the first month of the epidemic) were on average similarly profitable. The invariance of mean profit to fungicide sprays in response to disease reflects that disease occurred in relatively few yards and therefore fungicide sprays above the baseline were not triggered. In contrast, as the probability of pathogen overwintering increased from 0.001 to 0.1, the highest simulated profit occurred with approximately 4 to 6 fungicide applications in the first month of the epidemic. The consequences of making fewer applications were more pronounced when primary inoculum was more probable, and the most austere reductions in profit occurred when growers treated too little when disease was present. 

The number of fungicide applications that lead to the highest simulated profit depended on market conditions. Treating more often became more profitable as market demand decreased from high, to moderate, to low (Figure X, Y, Z). This outcome reflects the more stringent quality standards under low demand scenarios and thus the greater likelihood for crop devaluation due to quality defects in depressed market conditions. As quality expectations increased under depressed demand scenarios, mean grower profit was maximized by more intensive treatment that reduced costly quality-related defects.

\newpage

\begin{figure}[h]
    \centering
    \includegraphics[width=0.7\textwidth]{figures/2016_low.png}
    % \caption{}
    % \label{fig:fig2}
\end{figure}

\begin{figure}[h]
    \centering
    \includegraphics[width=0.7\textwidth]{figures/2016_moderate.png}
    % \caption{}
    % \label{fig:fig2}
\end{figure}

\begin{figure}[h]
    \centering
    \includegraphics[width=0.7\textwidth]{figures/2016_high.png}
    % \caption{}
    % \label{fig:fig2}
\end{figure}

\newpage

\begin{figure}[h]
    \centering
    \includegraphics[width=0.72\textwidth]{figures/2014_low.png}
    % \caption{}
    % \label{fig:fig2}
\end{figure}

\begin{figure}[h]
    \centering
    \includegraphics[width=0.72\textwidth]{figures/2014_moderate.png}
    % \caption{}
    % \label{fig:fig2}
\end{figure}

\begin{figure}[h]
    \centering
    \includegraphics[width=0.72\textwidth]{figures/2014_high.png}
    % \caption{}
    % \label{fig:fig2}
\end{figure}

\newpage

\subsubsection{Pathogen diversity}

In more complex scenarios where the pathogen population contained varying percentages of two strains that differed in the cultivars that each could infect, primary inoculum dose remained the dominant factor influencing mean profitability. Progressively more fungicide applications were needed to maximize mean profit as the pathogen population became dominated by the more virulent (\textit{V6}-virulent) strain (Figure X). Similarly to the simulations with primary inoculum dose, decreasing market demand increased the optimal number fungicide applications proportionately across the five levels of pathogen diversity we considered. 

\subsubsection{Primary inoculum location}

The location of where primary inoculum first occurred had small to non-discernible effects on the optimum fungicide application strategy or profit when primary inoculum dose was relatively low (i.e., XX). The greatest impact of where primary inoculum occurred was observed when inoculum dose was higher (probability 0.01 or 0.1). Whereas the optimum number of fungicide applications was little changed by the degree centrality of the yard where primary inoculum occurred, profit was reduced proportionately less with fewer fungicide applications in yards with sparse connections versus yards with the most connections. This effect was most notable under the high market demand scenario when crop devaluation from quality defects were least consequential.

\section{Results UPDATED}

The simulation results demonstrate a strong dependence of profitability on initial epidemic conditions, specifically highlighting the influence of initial inoculum dose, pathogen diversity, and the dispersal centrality of the initially infected yards on disease management strategies.

\subsection{Impact of Initial Inoculum Dose}

The initial probability of disease ($p_0$) emerged as a dominant factor influencing the profitability and efficacy of fungicide applications. 
- \textbf{Low Inoculum ($p_0 \leq 0.0001$):} Across all market demand scenarios, when the initial probability of disease was very low ($p_0 = 0.00001$ or $0.00005$), the relative change in profit remained close to zero regardless of the number of fungicide sprays in May. For example, in 2014 under low demand, the profit change was consistently between 0\% and $-1\%$ for all spray numbers and dispersal centrality percentiles.
- \textbf{Moderate to High Inoculum ($p_0 \geq 0.001$):} As $p_0$ increased, profitability became more sensitive to the number of sprays. For $p_0 = 0.005$ or $0.01$, failing to spray (0 sprays) resulted in profit losses of up to $-10\%$ (especially in high centrality yards and under high demand). In contrast, applying 4--6 sprays in May minimized losses, with profit changes typically between $-1\%$ and $-3\%$. For example, in the high demand scenario, at $p_0 = 0.01$, 0 sprays led to losses of $-10\%$ to $-12\%$ in the most central yards, while 5--6 sprays reduced losses to about $-2\%$ to $-3\%$.

\subsection{Effect of Pathogen Diversity}

Pathogen strain diversity also played an influential role, particularly in scenarios where the pathogen population included varying proportions of the \textit{V6}-virulent strain capable of infecting a broader range of cultivars. As the proportion of the \textit{V6}-virulent strain increased, the optimal number of fungicide sprays required to maximize profitability rose accordingly. When pathogen populations consisted predominantly or entirely of the \textit{V6}-virulent strain, fungicide applications became increasingly essential to maintaining profitability. Thus, pathogen diversity directly affected both the necessity and frequency of fungicide interventions, emphasizing the need for careful monitoring and targeted responses to pathogen strain composition.

Quantitatively, in the high demand scenario for $p_0 = 0.01$ and 100\% \textit{V6}, profit losses without spraying reached $-10\%$ to $-12\%$, while 5--6 sprays reduced losses to $-2\%$ to $-3\%$.

\subsection{Influence of Dispersal Centrality}

The location of initial infections, measured by the dispersal centrality of infected yards, significantly impacted profitability, particularly under high inoculum pressure scenarios. Initial infections in highly central yards were more detrimental, exacerbating the spread and subsequent economic losses across the region. For example, in the high demand scenario at $p_0 = 0.01$, 0 sprays led to losses of $-10\%$ to $-12\%$ in the most central yards, compared to $-6\%$ to $-8\%$ in the least central yards. In contrast, when initial infections occurred in less central yards, the relative economic losses were reduced, particularly under higher market demand scenarios. This finding underscores the strategic importance of identifying and rapidly addressing outbreaks in central locations within a network of interconnected agricultural units.

\subsection{Market Demand Conditions}

Different market demand conditions notably influenced the optimal disease management strategy. Under high market demand conditions, the impact of early fungicide interventions on profitability was moderate, reflecting more flexible quality standards. Conversely, under moderate and low market demand conditions, where quality requirements were more stringent, the profitability was much more sensitive to fungicide application strategies. Increased fungicide use was essential under these lower demand scenarios to prevent severe quality-related profit reductions. For instance, under low demand, the economic penalty for under-spraying was greatest, with losses up to $-10\%$ at high $p_0$ and 0 sprays, and only reduced to $-2\%$ to $-3\%$ with 5--6 sprays.

\textbf{Summary Table of Key Quantitative Results (2014, all demand scenarios):}

\begin{center}
\small
\begin{tabularx}{\textwidth}{lXXX}
\toprule
Scenario & 0 Sprays ($p_0=0.01$) & 5--6 Sprays ($p_0=0.01$) & 0 Sprays ($p_0=0.00001$) \\
\midrule
Low Demand, High Centrality & $-10\%$ & $-2\%$ & $0\%$ \\
Moderate Demand, High Centrality & $-8\%$ & $-2\%$ & $0\%$ \\
High Demand, High Centrality & $-12\%$ & $-3\%$ & $0\%$ \\
\bottomrule
\end{tabularx}
\end{center}

\textbf{Key Takeaways:}
- At very low initial inoculum, profit is nearly unaffected by spray regime.
- At high initial inoculum, 5--6 sprays in May are required to keep profit losses below $-3\%$.
- Losses are most severe in highly central yards and under high demand.
- Under-spraying at high $p_0$ can result in profit losses of $-8\%$ to $-12\%$.

\subsection{Comparison of Parameter Estimates Between May--June and June--July}

To better understand the temporal dynamics of the epidemic model, we compared the parameter estimates for the two key transition periods: May--June and June--July (see Appendix, Table~\ref{tab:parameter_estimates}).

Several notable differences emerge between these periods:

- The baseline log-odds of disease ($\beta_1$, $\beta_2$) are lower (more negative) in May--June ($-2.06$, $-4.15$) than in June--July ($-2.79$, $-3.88$), indicating a lower baseline risk of disease early in the season.
- The effect of autoinfection ($\delta_1$, $\delta_2$) is much larger in May--June ($2074.35$, $120.46$) compared to June--July ($2.94$, $8.04$), suggesting that local sources of inoculum are most influential at the start of the epidemic.
- The effect of dispersal from other yards ($\gamma_1$, $\gamma_2$) is also much greater in May--June ($34358.91$, $14536.03$) than in June--July ($1390.71$, $286.8$), highlighting the importance of regional spread early in the epidemic.
- The dispersal kernel parameters ($\alpha_1$, $\alpha_2$) are higher in June--July ($1.0$, $2.04$) than in May--June ($0.71$, $2.67$), indicating a possible shift in the spatial scale of dispersal as the season progresses.
- The fungicide effect parameters ($\eta_{11}$, $\eta_{12}$, $\eta_{21}$, $\eta_{22}$) are generally larger in May--June, especially $\eta_{11}$ ($3.51$ vs $0.02$), suggesting that fungicide applications are more effective at reducing disease early in the season.

Overall, these differences indicate that both local and regional sources of inoculum are most important in the early epidemic phase, and that fungicide applications have their greatest impact when applied early. As the season progresses, the epidemic becomes less sensitive to these factors, possibly due to depletion of susceptible hosts or environmental changes.

\section{Discussion}
Our motivating pathosystem represents a multidimensional and challenging problem for modeling, although the problem is not uncommon in agricultural systems with high-value crops. A highly dispersible and endemic pathogen has the potential to cause damage to both yield and quality. Market factors dictate what cultivars are produced, but market demand conditions dictate quality standards and the penalty for quality defects. Several strategies are available for disease mitigation: cultural controls that reduce primary inoculum; fungicide applications to reduce the rate of disease development; and strain-specific host resistance. 

Given these complexities, what strategies minimize economic losses industry wide?

Our analyses point to conditions in the earliest stages of epidemics as dominant factors that shape economic outcomes, with primary inoculum dose being of utmost importance. For scenarios where the probability of initial inoculum was not severely limiting to epidemic development (i.e., initial probability > 0.0001), the most influential factor controlling economic loss was the number of fungicide applications made in the first month of the epidemic. Pathogen strain diversity and the location of primary inoculum had smaller, moderating effects on economic losses. The relative importance of these effects were consistent across the three levels of market demand, although the optimal number of fungicide applications increased proportionately with more stringent quality standards in lower demand scenarios.  
 
Relative small changes in primary inoculum dose can have a dramatic effect on monocyclic and polycyclic epidemics \parencite{madden_study_2007}. For the later, this is particular apparent when dispersal characteristics of the pathogen produce disease gradients described by leptokurtic distributions \parencite{mundt_initial_2013, Xu1998}. Under such epidemics, the possibility of containment, and the efficiency of containment, is sensitive to primary inoculum dose, the number of contacts with infected individuals during the early stages of an epidemic, and the timing of intervention efforts \parencite{cunniffe_modeling_2016,danon_networks_2011,Severns2019}. This finding is not unexpected, but the sensitivity of optimal control strategies to primary inoculum dose is striking. An initial probability of overwintered inoculum of 0.0001 resulted in only minor economic damage, even in the absence of additional fungicide applications, independent of pathogen strain diversity or where the overwintered inoculum occurred. The blunting of the pathogen invasion hints at a critical pathogen density needed for initiating and persistence of epidemics of economic consequence. Pathogen threshold densities for epidemics in finite populations depend critically on the pathogen reproductive ratio and homogeneity of the host population \parencite{dallas_experimental_2018, Hartfield2013}. We also observed that epidemics caused only minor economic loss even when the pathogen population was composed entirely of \textit{V6}-virulent strains capable of infecting all cultivars, effectively creating a homogeneously susceptible host population.  
In the motivating hop powdery mildew pathosystem, the estimated frequency of overwintering inoculum was XXX in the raw data. Our analyses demonstrate that an approximately 10-fold reduction in initial inoculum dose could largely mitigate economic damage associated with fungicide inputs and crop damage. Such reductions are plausible with changes in spring cultural practices and planting of cultivars less susceptible to the disease \parencite{Laurie2023}, provided these changes are implemented on an area-wide basis \parencite{garcia-figuera_free-riding_2024, sherman_cooperation_2019}. When and where to deploy such changes for maximal effect will be the subject of future research. 

For a given level of demand and all but the lowest level of primary inoculum in our simulations, the economic consequence of epidemics are largely determined by the intensity of fungicide use. Both under treatment or over treatment could lead to sub-optimal economic outcomes, with under treatment generally leading to larger losses than over treatment in the present analyses. The overall economic damage of the disease become more costly as initial inoculum dose increased or virulence structure of the pathogen population enabled more yards to be infected. Under these conditions of increasing pathogen pressure, the optimal strategy for fungicide use is to treat more often (two to three times) in response to disease. The optimal strategy also became increasingly narrow and defined as disease affected more yards and thus deviation from the baseline fungicide strategy was necessary.

NOTE TO SELF: EXAMINE PARAMETER ESTIMATES. HOW MUCH OF FUNGICIDE EFFECT IS DUE TO DISEASE CONTROL IN TARGET YARD VS. POTENTIAL SOURCE YARDS?

Counter intuitively, depressed market conditions led to more intensive use of fungicides to minimize economic losses. Economic injury levels in pest management are intended to match the cost of control measures commensurate with the damage caused by a pest \parencite{stern_integration_1959}. Naively, one would expect that lower crop values would reduce the use of fungicides. We demonstrate the opposite can be true for a crop where quality standards are inexorably linked to price and demand. Under high pathogen pressure, the economic consequences of not treating become as costly as treating two to three-times more than necessary. 

It is clear that the results of the simulations are sensitive to the shape of the damage function, although the precise shape of this function is not known with certainty because of the vague language used in production contracts in the United States. Use of a single, fixed damage function would be appropriate in situations where quality standards are explicit and independent of demand, as may be appropriate in other production situations. We demonstrate that the absence of such standards may increase fungicide inputs because growers are incentivized to minimize quality defects and the associated risk of crop devaluation or rejection.

NOTE TO SELF: SECTION ON LOCATION OF INITIAL INFECTIONS AFTER ANALYSES ARE UPDATED




\section{Appendix}
We expand on the generalized regression model developed in \parencite{Gent2019a}. For a given yard $i$, we estimate the log-odds of disease $\eta_{i}$ given by

$$
\eta_{i, \tau}=\sum_{k=1}^{K} I_{k}^{(t)}(i)\left[\beta_{k, \tau}+\delta_{k, \tau}\left(\frac{\tilde{y}_{i}}{n_{\tilde{y}_{i}}} \exp{\left(-\eta_{1k} s_{i}\right)}\right)+\gamma_{k, \tau} \sum_{j=1}^{M_{i}}\left(\frac{a_{j} z_{j}}{n_{z_{j}}} \exp{\left(-\eta_{2k} s_{j}\right)} w_{i j} \exp{\left(-\alpha_{k, \tau} d_{i j}\right)} I_{k}^{(s)}(j)\right)\right]
$$

\subsection{Parameter estimates}


\begin{table}[ht]
\centering
\caption{Parameter estimates for epidemic model}
\label{tab:parameter_estimates}
\small
\begin{tabular}{llcc}
\toprule
\multicolumn{2}{l}{Parameter} & May--June & June--July \\
\midrule
\multicolumn{4}{l}{\textbf{Epidemic Model}} \\
$\beta_1$ & Baseline log-odds & $-2.06$ & $-2.79$ \\
$\beta_2$ & Baseline log-odds & $-4.15$ & $-3.88$ \\
$\delta_1$ & Autoinfection effect & $2074.35$ & $2.94$ \\
$\delta_2$ & Autoinfection effect & $120.46$ & $8.04$ \\
$\gamma_1$ & Dispersal effect & $34358.91$ & $1390.71$ \\
$\gamma_2$ & Dispersal effect & $14536.03$ & $286.8$ \\
$\alpha_1$ & Dispersal kernel & $0.71$ & $1.0$ \\
$\alpha_2$ & Dispersal kernel & $2.67$ & $2.04$ \\
$\eta_{11}$ & Fungicide effect & $3.51$ & $0.02$ \\
$\eta_{12}$ & Fungicide effect & $1.62$ & $0.31$ \\
$\eta_{21}$ & Fungicide effect & $2.9$ & $0.95$ \\
$\eta_{22}$ & Fungicide effect & $0.27$ & $0.37$ \\
\bottomrule
\end{tabular}
\end{table}


\begin{table}[ht]
\centering
\caption{Parameter estimates for damage functions}
\label{tab:damage_function_parameters}
\small
\begin{tabular}{llc}
\toprule
\multicolumn{2}{l}{Parameter} & Estimate \\
\midrule
\multicolumn{3}{l}{\textbf{Cone Incidence Model}} \\
 & $a_1$ & [value] \\
 & $b_1$ & [value] \\
\multicolumn{3}{l}{\textbf{Yield Loss Model}} \\
 & $a_2$ & [value] \\
\multicolumn{3}{l}{\textbf{Cone Color Model}} \\
 & $a_3$ & [value] \\
 & $b_3$ & [value] \\
\multicolumn{3}{l}{\textbf{Adjusted Price Model (Sigmoid Function)}} \\
\textit{High Demand} & $\theta_0$ & [value] \\
 & $\theta_1$ & [value] \\
\textit{Moderate Demand} & $\theta_0$ & [value] \\
 & $\theta_1$ & [value] \\
\textit{Low Demand} & $\theta_0$ & [value] \\
 & $\theta_1$ & [value] \\
\bottomrule
\end{tabular}
\end{table}

\begin{figure}[h]
    \centering
    \includegraphics[width=0.8\textwidth]{figures/graph_network_visual.png}
    \caption{}
    \label{fig:fig2}
\end{figure}

\newpage
\section*{Acknowledgements}

We thank the many individuals that provided technical support for this project and the participating grower cooperators that provide access to their fields and production records. This research was conducted in support of U.S. Department of Agriculture CRIS project 2072-21000-061-000-D. Funding was provided by National Institute of Food and Agriculture Specialty Crop Research Initiative award number (2021-51181-35901).

\printbibliography

\end{document}